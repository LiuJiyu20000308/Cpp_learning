\newpage

\section{类型推导}

\subsection{理解模板类型推导}
\label{sec:Item2-1}
\begin{verbatim}
template<typename T>
void f(ParamType param);
f(expr);
\end{verbatim}

在编译期间,编译器会用\texttt{expr}推到两个类型:\texttt{T}以及
\texttt{ParamType}。这两种类型通常是不一样的,因为\texttt{ParamType}经
常带修饰符,例如\texttt{const}或者\texttt{reference}。\texttt{T}类型的
推导不仅仅取决于\texttt{expr},也取决于\texttt{ParamType}的类型,分为
三种情况:
\begin{itemize}
\item \texttt{ParamType}是指针或者引用类型,但不是通用引用
  (Universal reference,\texttt{ParamType}带有\texttt{\&\&}修饰符,见
  Section~\ref{sec:ForwardingRef}).
\item \texttt{ParamType}是通用引用.
\item \texttt{ParamType}既不是指针也不是引用。
\end{itemize}

以下将分类进行讨论:
\begin{itemize}
\item 当\texttt{ParamType}是指针或者引用类型,但不是通用引用时,类型推
  导原理为:
    \begin{enumerate}
    \item 如果\texttt{expr}类型是引用的话,忽略引用。
    \item 通过令\texttt{expr}
    \end{enumerate}
    和\texttt{ParamType}类型一致来决定\texttt{T}。
  例如:
\begin{verbatim}
template<typename T>
void f(T& param);
void g(const T& param); // param is now a ref-to-const

int x = 27;        // x is an int
const int cx = x;  // cx is a const int
const int& rx = x; // rx is a reference to x as a const int

f(x);  // T is int, param's type is int&
f(cx); // T is const int, param's type is const int&
f(rx); // T is const int, param's type is const int&

g(x); // T is int, param's type is const int&
g(cx); // T is int, param's type is const int&
g(rx); // T is int, param's type is const int&
\end{verbatim}

  
\item \texttt{ParamType}为通用引用时,类型推导(见
  Section~\ref{sec:Item2-24})如下:
  \begin{enumerate}
  \item 如果\texttt{expr}为左值,则\texttt{T}和\texttt{ParamType}均为
    左值引用,\textbf{这是唯一一种\texttt{T}被推导为引用的情况},另外
    值得注意的是,虽然\texttt{ParamType}带有\texttt{\&\&},\textbf{但
      还是被推导为左值引用。}
  \item 如果\texttt{expr}为右值,那么推导和第一种情况类似。
  \end{enumerate}

  例如:
\begin{verbatim}
template<typename T>
void f(T&& param); // param is now a universal reference

int x = 27; // as before
const int cx = x; // as before
const int& rx = x; // as before

f(x);  // x is lvalue, so T is int&,param's type is also int&
f(cx); // cx is lvalue, so T is const int&, param's type is also const int&
f(rx); // rx is lvalue, so T is const int&, param's type is also const int&
f(27); // 27 is rvalue, so T is int, param's type is therefore int&&
\end{verbatim}

\item \texttt{ParamType}既不是指针也不是引用时,这意味着\texttt{param}
  将会传递输入参数的副本,因此此时的类型推导如下:
  \begin{enumerate}
  \item 如果\texttt{expr}是一个引用,则忽略引用部分。
  \item 如果忽略引用之后是\texttt{const}的,也将\texttt{const}忽略。如
    果是\texttt{volatile}(见Section~\ref{sec:Item2-40})的也要忽略。
  \end{enumerate}
  例如:
\begin{verbatim}
template<typename T>
void f(T param); // param is now passed by value

int x = 27;       
const int cx = x; 
const int& rx = x; 
f(x);  // T's and param's types are both int
f(cx); // T's and param's types are again both int
f(rx); // T's and param's types are still both int
\end{verbatim}
  这样做的原因是\texttt{cx}和\texttt{rx}不能被更改不代表\texttt{param}
  不能被更改,另外,要意识到\textbf{\texttt{const}和\texttt{volatile}
    只有在按值传入参数的时候会被忽略。}

  如果\texttt{expr}是一个指向\texttt{const}对象的\texttt{const}指针的
  话,\textbf{指针的\texttt{const}会被忽略}:
\begin{verbatim}
// ptr is const pointer to const object
const char* const ptr = "Fun with pointers";

// type deduced for param will be const char*
f(ptr);
\end{verbatim}
\end{itemize}

\textbf{在许多情况下,数组会退化为指向第一个元素的指针。当数组被传入到
  一个by-value参数的模板中呢?}

\begin{verbatim}
template<typename T>
void f(T param); // template with by-value parameter

const char name[] = "J. P. Briggs"; // name's type is const char[13]
f(name); // name is array, but T deduced as const char*
\end{verbatim}

但是如果\texttt{expr}是一个引用呢?

\begin{verbatim}
template<typename T>
void f(T& param); // template with by-reference parameter

f(name); // T is deduced to be const char [13]
//paramType is const char (&)[13]
\end{verbatim}

这种语法使得我们可以利用模板来推导出数组元素个数:
\begin{verbatim}
template<typename T, std::size_t N> 
constexpr std::size_t arraySize(T (&)[N]) noexcept{
  return N; 
}
\end{verbatim}

将函数声明为\texttt{constexpr}可以让结果在编译器就可以使用,同时也能让
编译器生成更好的代码(见Section \ref{sec:Item2-14}, \ref{sec:Item2-15}):
\begin{verbatim}
int keyVals[] = { 1, 3, 7, 9, 11, 22, 35 }; // keyVals has 7 elements
int mappedVals[arraySize(keyVals)]; // so does mappedVals
std::array<int, arraySize(keyVals)> mappedVals; // mappedVals'size is 7 
\end{verbatim}

同样的,\textbf{函数类型也会退化到function指针}:

\begin{verbatim}
void someFunc(int, double); // someFunc is a function with type
void(int, double)

template<typename T>
void f1(T param); // in f1, param passed by value
template<typename T>
void f2(T& param); // in f2, param passed by ref

f1(someFunc); // param deduced as ptr-to-func;
// type is void (*)(int, double)
f2(someFunc); // param deduced as ref-to-func;
// type is void (&)(int, double)
\end{verbatim}

\subsection{理解auto类型推导}
\label{sec:Item2-2}

当一个变量用\texttt{auto}声明的时候,\texttt{auto}扮演模板中的
\texttt{T}的作用,变量的类型修饰符可以看做\texttt{ParamType}。

同样的,\texttt{auto}也分为三个情况:
\begin{itemize}
\item 类型修饰符是指针或引用,但不是通用引用。
\item 类型修饰符是通用引用。
\item 类型修饰符既不是指针也不是引用。
\end{itemize}

\begin{verbatim}
auto x = 27;        // case 3 (x is neither ptr nor ref)
const auto cx = x;  // case 3 (cx isn't either)
const auto& rx = x; // case 1 (rx is a non-universal ref)

auto&& uref1 = x;  // uref1's type is int&
auto&& uref2 = cx; // uref2's type is const int&
auto&& uref3 = 27; // uref3's type is int&&

const char name[] = "R. N. Briggs"; // name's type is const char[13]
auto arr1 = name; // arr1's type is const char*
auto& arr2 = name; // arr2's type is const char (&)[13]

void someFunc(int, double); //void(int, double)
auto func1 = someFunc; // func1's type is void (*)(int, double)
auto& func2 = someFunc; // func2's type is void (&)(int, double)
\end{verbatim}
可以看到,\texttt{auto}类型推导和模板类型推导一致。只有一种情况下,它
们是不同的:

\begin{verbatim}
int x1 = 27;
int x2(27);
int x3 = { 27 };
int x4{ 27 };
\end{verbatim}
它们都是一个结果:值为27的\texttt{int},但是如果我们换成\texttt{auto}
的话,对于后两个而言,会定义一个只包含一个元素27的类型为
\texttt{std::initializer\_list<int>}的对象。
\begin{verbatim}
auto x1 = 27; // type is int, value is 27
auto x2(27); // ditto
auto x3 = { 27 }; // type is std::initializer_list<int>, value is { 27 }
auto x4{ 27 }; // ditto
\end{verbatim}

花括号初始化的时候,\texttt{auto}会推断为
\texttt{std::initializer\_list<int>},假如花括号内部类型不相同,则会报
错:
\begin{verbatim}
auto x5 = { 1, 2, 3.0 }; // error! can't deduce T for std::initializer_list<T>
\end{verbatim}

然而对于模板而言,使用花括号的话类型推导会失败:
\begin{verbatim}
auto x = { 11, 23, 9 };
template<typename T> 
void f(T param);
f({ 11, 23, 9 }); // error! can't deduce type for T

template<typename T>
void f(std::initializer_list<T> initList);
f({ 11, 23, 9 });
\end{verbatim}

C++14中\texttt{auto}可以用在函数返回类型中,也可以在lambda表达式的参数
声明中使用,然而\textbf{它们使用的是模板类型推导而不是\texttt{auto}类
  型推导。}

\begin{verbatim}
auto createInitList(){
  return { 1, 2, 3 }; // error: can't deduce type.
}

std::vector<int> v;
auto resetV = [&v](const auto& newValue) { v = newValue; }; 
resetV({ 1, 2, 3 }); // error! can't deduce type.
\end{verbatim}


\subsection{理解decltype}
\label{sec:Item2-3}

\texttt{decltype}会告诉你表达式的类型:
\begin{verbatim}
const int i = 0; // decltype(i) is const int
bool f(const Widget& w); // decltype(w) is const Widget&
// decltype(f) is bool(const Widget&)

struct Point {
int x, y; // decltype(Point::x) is int
}; // decltype(Point::y) is int
Widget w; // decltype(w) is Widget

if (f(w)) … // decltype(f(w)) is bool

template<typename T> // simplified version of std::vector
class vector {
public:
…
T& operator[](std::size_t index);
…
};
vector<int> v; // decltype(v) is vector<int>
if (v[0] == 0) … // decltype(v[0]) is int&
\end{verbatim}

在C++11中\texttt{decltype}主要用来声明那些返回值依赖于参数类型的函数模
板。例如,假如函数要返回\texttt{vector<bool>}中的一个元素,然而
\texttt{operator[]}并不会返回\texttt{bool\&}而是一个新的object,这里的
关键是返回值是依赖于容器的:
\begin{verbatim}
// works, but requires refinement.
template<typename Container, typename Index> 
auto authAndAccess(Container& c, Index i) -> decltype(c[i]){
  authenticateUser();
  return c[i];
}
\end{verbatim}
这就是C++11中的\textbf{返回值类型后置语法}。在C++14中可以只留下
\texttt{auto}:
\begin{verbatim}
template<typename Container, typename Index>
auto authAndAccess(Container& c, Index i){
  authenticateUser();
  return c[i]; // return type deduced from c[i]
}
\end{verbatim}

这里\texttt{auto}按照模板类型推导,然而这里是存在问题的,虽然
\texttt{operator[]}返回\texttt{T\&},但是在推导时引用会被忽略,因此下
面的代码是无法编译的:
\begin{verbatim}
std::deque<int> d;
authAndAccess(d, 5) = 10; 
\end{verbatim}

C++14中,我们可以这样改进,我们称之为\texttt{decltype}推导方法:
\begin{verbatim}
// still have refinement.
template<typename Container, typename Index> 
decltype(auto) authAndAccess(Container& c, Index i){
  authenticateUser();
  return c[i];
} // return type T&.
\end{verbatim}

这种推导方式也可以用作初始化表达式:
\begin{verbatim}
Widget w;
const Widget& cw = w;
auto myWidget1 = cw; // myWidget1's type is Widget
decltype(auto) myWidget2 = cw; // myWidget2's type is const Widget&
\end{verbatim}

上面说到的refinement实际上是说,如果我想传入一个右值,例如一个临时的
Container,这是无法编译通过的,因此我们还可以做出改进:

\begin{verbatim}
template<typename Container, typename Index> // final C++14 version
decltype(auto) authAndAccess(Container&& c, Index i){
  authenticateUser();
  return std::forward<Container>(c)[i];
}

template<typename Container, typename Index> // final C++11 version
auto authAndAccess(Container&& c, Index i)
-> decltype(std::forward<Container>(c)[i]){
  authenticateUser();
  return std::forward<Container>(c)[i];
}
\end{verbatim}
这里我们使用了\texttt{std::forward},参见
Section~\ref{sec:ForwardingRef}.

当然,在极少数情况下\texttt{decltype}会得不到你期望的类型:如果对一个
名称使用\texttt{decltype}是没有问题的,但是如果对于一个比名称复杂的左
值表达式,\texttt{decltype}都会返回左值引用:
\begin{verbatim}
decltype(auto) f1(){
  int x = 0;
  return x; // decltype(x) is int, so f1 returns int
}
decltype(auto) f2(){
  int x = 0;
  return (x); // decltype((x)) is int&, so f2 returns int&
}
\end{verbatim}

\subsection{了解如何去看类型推导}
\label{sec:Item2-4}

有以下几种方法:
\begin{itemize}
\item 通过IDE编辑器来看。
\item 通过编译器诊断:
\begin{verbatim}
template<typename T> // declaration only for TD;
class TD; // TD == "Type Displayer"
TD<decltype(x)> xType;
\end{verbatim}
  编译器会报错:
\begin{verbatim}
error: 'xType' uses undefined class 'TD<int>'
\end{verbatim}
\item 运行期也可以输出:
\begin{verbatim}
#include <boost/type_index.hpp>
std::cout << typeid(x).name() << '\n'; // display types for x
using boost::typeindex::type_id_with_cvr;
cout << "T = " << type_id_with_cvr<T>().pretty_name();
\end{verbatim}
\end{itemize}
%%% Local Variables:
%%% mode: latex
%%% TeX-master: "../CppLearning"
%%% End:
